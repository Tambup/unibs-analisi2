\chapter{Temi Esame}
\section{T.E. 2012/2013 scritto n.1}
\begin{exercise}
	Sia $f: \R^2 \rightarrow \R$ data da $f(x, y)=e^{-\abs{4\cdot arctan(x\cdot y^2)}}$
	\begin{description}
		\item[A] Nessuna delle altre affermazioni è esatta
		\item[B] $f$ ammette almeno un punto di minimo assoluto
		\item[C] $\inf_R^2f = 0$
		\item[D] $f$ ha infiniti punti di massimo
	\end{description}
	L'esponensiale è una funzione monotona crescente quindi la ricerca di massimi a minimi si sposta alla ricerca dei massimi e minimi dell'esponente.\\
	L'esponente assume sempre valori negativi. Inotre risulta essere una quantità limitata tra $[0;4\frac{\pi}{0}[$, quindi $\sup_R^2f=e^0=1$ e $\inf_R^2f=e^{-2\pi}$\\
	Sono quindi punti di massimo tutti i punti che rendono nullo l'esponente: $arctan(xy^2)=0\Rightarrow x=0,\forall y or y=0,\forall x$ che sono i due assi. Essendo questi punti del dominio allora si può dire $\sup_R^2f=\max_R^2f=0$\\
	I punti di minimo si hanno per $\abs{arctan(xy^2)}=\frac{\pi}{2}$ quindi per $x\rightarrow \pm\infty$ or $y\rightarrow \pm\infty$ essendo questi valori al limite il valore $e^{-2\pi}$ è $inf$ per $f$\\
	La risposta vera è quindi la D.\\
\end{exercise}
\begin{exercise}
	Sia $(X,d)$ uno spazio metrico e siano $A, B$ sottoinsiemi di $X$. Quale/i delle seguenti affermazioni è/sono certamente vera/e?
	\begin{description}
		\item[1] $A\subseteq B\Rightarrow\partial A\subseteq \partial B$
		\item[2] $A\subseteq B\Rightarrow\overline{A}\subseteq\overline{B}$
	\end{description}
	\begin{description}
		\item[A] Entrambe
		\item[B] Solo la seconda
		\item[C] Nessuna delle affermazioni è esatta
		\item[D] Solo la prima
	\end{description}
	La prima affermazione è certamente falsa poiché se scelto come spazio metrico $R^2$ con distanza quella eclidea. Scelgo $A=B((0,0),2), A=B((0,0),1)$ allora si ha che $\partial A = \{(x,y)\in \R^2:d((x,y),(0,0))=2\}$ e $\partial B = \{(x,y)\in \R^2:d((x,y),(0,0))=1\}$ e questi due insiemi sono disgiunti.\\
	la seconda è vera ma devo pensarci un po...\\
\end{exercise}


